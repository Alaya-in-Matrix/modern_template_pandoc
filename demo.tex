\documentclass[italian,ignorenonframetext,]{beamer}
\setbeamertemplate{caption}[numbered]
\setbeamertemplate{caption label separator}{: }
\setbeamercolor{caption name}{fg=normal text.fg}
\beamertemplatenavigationsymbolsempty
\usepackage{lmodern}
\usepackage{amssymb,amsmath}
\usepackage{ifxetex,ifluatex}
\usepackage{fixltx2e} % provides \textsubscript
\ifnum 0\ifxetex 1\fi\ifluatex 1\fi=0 % if pdftex
  \usepackage[T1]{fontenc}
  \usepackage[utf8]{inputenc}
\else % if luatex or xelatex
  \ifxetex
    \usepackage{mathspec}
  \else
    \usepackage{fontspec}
  \fi
  \defaultfontfeatures{Ligatures=TeX,Scale=MatchLowercase}
    \setmainfont[]{DejaVu Sans}
    \setsansfont[]{PT Sans}
    \setmonofont[Mapping=tex-ansi]{DejaVu Sans Mono}
\fi
\usetheme{m}
\usefonttheme{serif} % use mainfont rather than sansfont for slide text
% use upquote if available, for straight quotes in verbatim environments
\IfFileExists{upquote.sty}{\usepackage{upquote}}{}
% use microtype if available
\IfFileExists{microtype.sty}{%
\usepackage{microtype}
\UseMicrotypeSet[protrusion]{basicmath} % disable protrusion for tt fonts
}{}
\ifnum 0\ifxetex 1\fi\ifluatex 1\fi=0 % if pdftex
  \usepackage[shorthands=off,main=italian]{babel}
\else
  \usepackage{polyglossia}
  \setmainlanguage[]{italian}
\fi
\newif\ifbibliography
\usepackage{color}
\usepackage{fancyvrb}
\newcommand{\VerbBar}{|}
\newcommand{\VERB}{\Verb[commandchars=\\\{\}]}
\DefineVerbatimEnvironment{Highlighting}{Verbatim}{commandchars=\\\{\}}
% Add ',fontsize=\small' for more characters per line
\usepackage{framed}
\definecolor{shadecolor}{RGB}{42,33,28}
\newenvironment{Shaded}{\begin{snugshade}}{\end{snugshade}}
\newcommand{\KeywordTok}[1]{\textcolor[rgb]{0.26,0.66,0.93}{\textbf{{#1}}}}
\newcommand{\DataTypeTok}[1]{\textcolor[rgb]{0.74,0.68,0.62}{\underline{{#1}}}}
\newcommand{\DecValTok}[1]{\textcolor[rgb]{0.27,0.67,0.26}{{#1}}}
\newcommand{\BaseNTok}[1]{\textcolor[rgb]{0.27,0.67,0.26}{{#1}}}
\newcommand{\FloatTok}[1]{\textcolor[rgb]{0.27,0.67,0.26}{{#1}}}
\newcommand{\ConstantTok}[1]{\textcolor[rgb]{0.74,0.68,0.62}{{#1}}}
\newcommand{\CharTok}[1]{\textcolor[rgb]{0.02,0.61,0.04}{{#1}}}
\newcommand{\SpecialCharTok}[1]{\textcolor[rgb]{0.02,0.61,0.04}{{#1}}}
\newcommand{\StringTok}[1]{\textcolor[rgb]{0.02,0.61,0.04}{{#1}}}
\newcommand{\VerbatimStringTok}[1]{\textcolor[rgb]{0.02,0.61,0.04}{{#1}}}
\newcommand{\SpecialStringTok}[1]{\textcolor[rgb]{0.02,0.61,0.04}{{#1}}}
\newcommand{\ImportTok}[1]{\textcolor[rgb]{0.74,0.68,0.62}{{#1}}}
\newcommand{\CommentTok}[1]{\textcolor[rgb]{0.00,0.40,1.00}{\textit{{#1}}}}
\newcommand{\DocumentationTok}[1]{\textcolor[rgb]{0.00,0.40,1.00}{\textit{{#1}}}}
\newcommand{\AnnotationTok}[1]{\textcolor[rgb]{0.00,0.40,1.00}{\textbf{\textit{{#1}}}}}
\newcommand{\CommentVarTok}[1]{\textcolor[rgb]{0.74,0.68,0.62}{{#1}}}
\newcommand{\OtherTok}[1]{\textcolor[rgb]{0.74,0.68,0.62}{{#1}}}
\newcommand{\FunctionTok}[1]{\textcolor[rgb]{1.00,0.58,0.35}{\textbf{{#1}}}}
\newcommand{\VariableTok}[1]{\textcolor[rgb]{0.74,0.68,0.62}{{#1}}}
\newcommand{\ControlFlowTok}[1]{\textcolor[rgb]{0.26,0.66,0.93}{\textbf{{#1}}}}
\newcommand{\OperatorTok}[1]{\textcolor[rgb]{0.74,0.68,0.62}{{#1}}}
\newcommand{\BuiltInTok}[1]{\textcolor[rgb]{0.74,0.68,0.62}{{#1}}}
\newcommand{\ExtensionTok}[1]{\textcolor[rgb]{0.74,0.68,0.62}{{#1}}}
\newcommand{\PreprocessorTok}[1]{\textcolor[rgb]{0.74,0.68,0.62}{\textbf{{#1}}}}
\newcommand{\AttributeTok}[1]{\textcolor[rgb]{0.74,0.68,0.62}{{#1}}}
\newcommand{\RegionMarkerTok}[1]{\textcolor[rgb]{0.74,0.68,0.62}{{#1}}}
\newcommand{\InformationTok}[1]{\textcolor[rgb]{0.00,0.40,1.00}{\textbf{\textit{{#1}}}}}
\newcommand{\WarningTok}[1]{\textcolor[rgb]{1.00,1.00,0.00}{\textbf{{#1}}}}
\newcommand{\AlertTok}[1]{\textcolor[rgb]{1.00,1.00,0.00}{{#1}}}
\newcommand{\ErrorTok}[1]{\textcolor[rgb]{1.00,1.00,0.00}{\textbf{{#1}}}}
\newcommand{\NormalTok}[1]{\textcolor[rgb]{0.74,0.68,0.62}{{#1}}}

% Prevent slide breaks in the middle of a paragraph:
\widowpenalties 1 10000
\raggedbottom

\AtBeginPart{
  \let\insertpartnumber\relax
  \let\partname\relax
  \frame{\partpage}
}
\AtBeginSection{
  \ifbibliography
  \else
    \let\insertsectionnumber\relax
    \let\sectionname\relax
    \frame{\sectionpage}
  \fi
}
\AtBeginSubsection{
  \let\insertsubsectionnumber\relax
  \let\subsectionname\relax
  \frame{\subsectionpage}
}

\setlength{\emergencystretch}{3em}  % prevent overfull lines
\providecommand{\tightlist}{%
  \setlength{\itemsep}{0pt}\setlength{\parskip}{0pt}}
\setcounter{secnumdepth}{0}
%%
\metroset{numbering=fraction,
            progressbar=frametitle,
            background=light,
            titleformat=uppercase,
            frametitleformat=uppercase,
            sectiontitleformat=regular,
            block=fill}

\title{Lezione Demo}
\author{Davide Parisi}
\institute{Università di Bari - ``Aldo Moro''}
\date{\today}

\begin{document}
\frame{\titlepage}

\section{Parte prima}\label{parte-prima}

\begin{frame}{Introduzione}

Lorem ipsum dolor sit amet, consectetur adipisicing elit, sed do eiusmod
tempor incididunt ut labore et dolore magna aliqua. Ut enim ad minim
veniam, quis nostrud exercitation ullamco laboris nisi ut aliquip ex ea
commodo consequat. Duis aute irure dolor in reprehenderit in voluptate
velit esse cillum dolore eu fugiat nulla pariatur. Excepteur sint
occaecat cupidatat non proident, sunt in culpa qui officia deserunt
mollit anim id est laborum.

\end{frame}

\begin{frame}{Situazione attuale}

Lorem ipsum \alert{dolor} sit amet, consectetur adipisicing elit, sed do
eiusmod tempor incididunt ut labore et dolore magna aliqua.

\begin{alertblock}{alert}
  alert Block
\end{alertblock}

\metroset{block=fill}

\begin{exampleblock}{Example}
  example Block
\end{exampleblock}

\end{frame}

\begin{frame}{Concetti base}

\begin{itemize}
\tightlist
\item
  uno
\item
  due
\item
  tre
\end{itemize}

\end{frame}

\begin{frame}{Concetti base}

\begin{enumerate}
\def\labelenumi{\arabic{enumi}.}
\tightlist
\item
  uno
\item
  due

  \begin{enumerate}
  \def\labelenumii{\arabic{enumii}.}
  \tightlist
  \item
    duepuntouno
  \item
    duepuntodue
  \end{enumerate}
\item
  tre
\end{enumerate}

\end{frame}

\section{Seconda Parte}\label{seconda-parte}

\begin{frame}{Matematica}

\[X_i + Y_i = A_i\]

\end{frame}

\begin{frame}{Quotes}

\begin{quote}
Lorem ipsum dolor sit amet

-- someone
\end{quote}

\end{frame}

\begin{frame}[fragile]{Code}

\metroset{block=fill}

\begin{exampleblock}{\textbf{Ruby example:}}
Produce: "hello"
\end{exampleblock}

\begin{Shaded}
\begin{Highlighting}[]
\KeywordTok{def} \NormalTok{hello}
  \NormalTok{puts }\StringTok{'hello'}
\KeywordTok{end}
\end{Highlighting}
\end{Shaded}

\begin{exampleblock}{\textbf{PHP example:}}
  Produce: "Hello World!"
\end{exampleblock}

\begin{Shaded}
\begin{Highlighting}[]
\KeywordTok{<?php}
  \FunctionTok{echo} \StringTok{"Hello World!"}\OtherTok{;}
\KeywordTok{?>}
\end{Highlighting}
\end{Shaded}

\end{frame}

\begin{frame}[fragile]{Code}

\begin{exampleblock}{Java code example}
  Main method example
\end{exampleblock}

\begin{Shaded}
\begin{Highlighting}[]
\KeywordTok{public} \DataTypeTok{void} \FunctionTok{main}\NormalTok{(String[] argv) \{}
    \CommentTok{//TODO: your code here}
\NormalTok{\}}
\end{Highlighting}
\end{Shaded}

\end{frame}

\end{document}
